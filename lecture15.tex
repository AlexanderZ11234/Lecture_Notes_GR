\section{Lecture 15: Einstein gravity}

Recall that in Newtonian spacetime, we were able to reformulate the Poisson law $\Delta \phi = 4\pi G_N \rho$ in terms of the Newtonian spacetime curvature as 

\[
R_{00} = 4\pi G_N \rho
\]
$R_{00}$ with respect to $\nabla_{\text{Newton}}$

$G_N = $ Newtonian gravitational constant

This prompted Einstein to postulate $<$ 1915 that the relativistic field equations for the Lorentzian metric $g$ of (relativistic) spacetime
\[
R_{ab} = 8\pi G_N T_{ab} \cancel{ \quad }
\]

However, this equation suffers from a problem

LHS:
$(\nabla_a R)^{ab} \neq 0$ \\
generically

RHS:
\[
(\nabla_a T)^{ab}  = 0
\]
thought bubble: $=$ formulated from an action

Einstein tried to argue this problem away.  

Nevertheless, the equations cannot be upheld. 

\subsection{Hilbert}

Hilbert was a specialist for variational principles. 

To find the appropriate left hand side of the gravitational field equations, Hibert suggested to start from an action
\[
S_{\text{Hilbert}}[g] = \int_M \sqrt{-g} R_{ab}g^{ab}
\]
thought bubble $=$ ``simplest action''

\underline{aim}: varying this w.r.t. metric $g_{ab}$ will result in some tensor 
\[
G^{ab} = 0
\]

\subsection{Variation of $S_{\text{Hilbert}}$}

\[
\begin{gathered}
  0 \overset{!}{=} \underbrace{\delta}_{g_i} S_{\text{Hilbert}}[g] = \int_M [ \underbrace{ \delta \sqrt{-g} g^{ab}R_{ab} }_{1} + \underbrace{ \sqrt{-g} \delta g^{ab} R_{ab}}_{2} + \underbrace{ \sqrt{-g} g^{ab} \delta R_{ab} }_{3} ] 
\end{gathered}
\]
\[
\text{ and 1 : } \delta \sqrt{-g} = \frac{ - (\text{det}g)g^{mn} \delta g_{mn} }{ 2 \sqrt{-g}} = \frac{1}{2} \sqrt{-g} g^{mn} \delta g_{mn}
\]
thought bubble
\[
\begin{gathered}
  \delta \text{det}(g) = \text{det}(g) g^{mn} \delta g_{mn} \\ 
  \text{ e.g. from } \\
\text{det}(g) = \exp{ \text{tr}{ \ln{g} } }
\end{gathered}
\]

ad 2: $g^{ab}g_{bc} = \delta^a_c$
\[
\begin{gathered}
\Longrightarrow (\delta g^{ab})g_{bc} + g^{ab}(\delta g_{bc}) = 0  \\
 \Longrightarrow \delta g^{ab} = -g^{am} g^{bn} \delta g_{mn}
\end{gathered}
\]

ad 3: 
\[
\begin{gathered}
  \Delta R_{ab} \underbrace{=}_{\text{normal coords at point}} \delta \partial_b \Gamma^m_{ \, \, am} - \delta \partial_m \Gamma^m_{ \, \, ab} + \Gamma \Gamma - \Gamma \Gamma = \\
  \begin{aligned}
    & = \partial_b \delta \Gamma^m_{ \, \, am} - \partial_m \delta \Gamma^m_{ \, \, ab } = \\  
    & = \nabla_b (\delta \Gamma)^m_{ \, \, am} - \nabla_m (\delta \Gamma)^m_{ \, \, ab}
\end{aligned} \\
\Longrightarrow \sqrt{-g} g^{ab} \delta R_{ab} = \sqrt{-g}
\end{gathered}
\]
``if you formulate the variation properly, you'll see the variation $\delta$ commute with $\partial _b$'' EY : 20150408 I think one uses the integration at the bounds, integration by parts trick

$\Gamma^i_{(x) \, \, jk } - \widetilde{\Gamma}^i_{ (x) \, \, jk }$ are the components of a $(1,2)$-tensor.

Notation: $(\nabla_b A)^i_{ \, \, g} =: A^i_{ \, \, j;b}$

\[
\begin{gathered}
\Longrightarrow \sqrt{-g} g^{ab} \delta R_{ab}  \\
\underbrace{=}_{ \nabla g = 0 } \sqrt{-g} (g^{ab} \delta \Gamma^m_{ \, \, am} )_{;b} - \sqrt{-g} (g^{ab} \delta \Gamma^m_{ \, \, ab} )_{ ; m} = \sqrt{-g} A^b_{ \, \, ; b} - \sqrt{-g} B^m_{ \, \, , m }
\end{gathered}
\]

Question: Why is the difference of coefficients a tensor?

Answer:
\[
\begin{aligned}
\Gamma_{(y) \, \, jk}^i = \frac{ \partial y^i}{ \partial x^m} \frac{ \partial x^m}{ \partial y^j} \frac{ \partial x^q}{ \partial y^k} \Gamma^m_{(x) \,\ , nq} + \frac{ \partial y^i}{ \partial x^m} \frac{ \partial^2 x^m}{ \partial y^j \partial y^k}
\end{aligned}
\]

Collecting terms, one obtains

\[
\begin{aligned}
  0 & \overset{!}{=} \delta S_{\text{Hilbert}} = \int_M [ \frac{1}{2} \sqrt{-g} g^{mn} \delta g_{mn} g^{ab} R_{ab} - \sqrt{-g} g^{am} g^{bn} \delta g_{mn} R_{ab}+    \underbrace{ (\sqrt{-g}A^a)_{ \, , a} }_{ \text{surface} } - \underbrace{ ( \sqrt{-g} B^b)_{ \, , b } }_{ \text{surface term } } ] \\
  & = \int_M \sqrt{-g} \delta \underbrace{g_{mn}}_{ \text{arbitrary variation}} [ \frac{1}{2} g^{mn} R - R^{mn} ] \Longrightarrow G^{mn} = R^{mn} - \frac{1}{2} g^{mn} R
\end{aligned}
\]

Hence Hilbert, from this ``mathematical'' argument, concluded that one may take
\[
\begin{gathered}
\boxed{ R_{ab} - \frac{1}{2} g_{ab} R = 8 \pi G_N T_{ab} }  \\
 \text{ Einstein equations}
\end{gathered}
\]
\[
S_{E-H}[g] = \int_M \sqrt{-g}R
\]

\subsection{Solution of the $\nabla_a T^{ab} =0$ issue}

One can show ($\to$ Tutorials) that the \underline{Einstein curvature}
\[
G_{ab} = R_{ab} - \frac{1}{2} g_{ab}R
\]
satisfy the so-called \underline{contracted} \underline{differential Bianchi identity}
\[
(\nabla_a G)^{ab} =0 
\]

\subsection{Variants of the field equations}

\begin{enumerate}
\item[(a)] a simple rewriting:
\[
R_{ab} - \frac{1}{2} g_{ab} R = 8 \pi G_N T_{ab} = T_{ab}
\]
$G_N = \frac{1}{8\pi}$

Contract on both sides $g^{ab}$

\[
\begin{gathered}
\begin{aligned}
  & R_{ab} - \frac{1}{2} g_{ab} R = T_{ab} || g^{ab} \\ 
  & R - 2R = T := T_{ab}g^{ab}
\end{aligned} \\
\Longrightarrow R = -T
\end{gathered}
\]
\[
\begin{gathered}
\Longrightarrow R_{ab} + \frac{1}{2} g_{ab} T = T_{ab} \\
\Longleftrightarrow R_{ab} = (T_{ab} - \frac{1}{2} Tg_{ab}) =: \widehat{T}_{ab}
\end{gathered}
\]
\[
\boxed{ R_{ab} = \widehat{T}_{ab}}
\]

\item[(b)] \[
S_{E-H}[g] := \int_M \sqrt{-g} (R+ 2\Lambda)
\]
thought bubble: $\Lambda$ cosmological constant

\underline{History:}

1915: $\Lambda < 0$ (Einstein) in order to get a non-expanding universe

$>$1915: $\Lambda =0$ Hubble 

today $\Lambda > 0$ to account for an accelerated expansion

$\Lambda \neq 0$ can be interpreted as a contribution

$-\frac{1}{2} \Lambda g$ to the energy-momentum 

``dark energy''

Question: surface terms scalar?

Answer: for a careful treatment of the surface terms which we discarded, see, e.g. E. Poisson, ``A relativist's toolkit'' C.U.P. ``excellent book''

Question: What is a constant on a manifold?

Answer: $\int \sqrt{-g} \Lambda = \Lambda \int \sqrt{-g} 1$

[back to dark energy]

[Weinberg, QCD, calculated] \\
\underline{idea}: 1 could arise as the vacuum energy of the standard model fields 

$\Lambda_{\text{calculated}} = 10^{120} \times \Lambda_{\text{obs}}$

``worst prediction of physics''

\underline{Tutorials}: \underline{check that }
\begin{itemize}
\item Schwarzscheld metric (1916)
\item FRW metric 
\item pp-wave metric 
\item Reisner-Nordstrom 
\end{itemize}
$\Longrightarrow $ are solutions to Einstein's equations
\end{enumerate}

in high school 

$m\ddot{x} + m\omega^2 x^2=0$

$x(t) = \cos{(\omega t)}$

\underline{ET}: [elementary tutorials]

study motion of particles \& observers in Schwarzscheld S.T.

\underline{Satellite}: Marcus C. Werner

Gravitational lensing

odd number of pictures Morse theory (EY:20150408 Morse Theory !!!)

\underline{Domenico Giulini}

Hamiltonian form
Canonical Formulations

Key to Quantum Gravity
