\subsection*{Tutorial 11 Symmetry }

\exercisehead{1}\textbf{: True or false?}

\begin{enumerate}
\item[(a)]
\begin{itemize}
\item
\item $\phi^*:T^*N \to T*M$ i.e. $\phi^*\nu(X) = \nu(\phi_*X)$ for smooth $\phi:M \to N$, so the pullback of a covector $\nu \in T^*N$ maps to a covector in $T*M$.  
\item
\item
\item
\item
\end{itemize}
\item[(b)]
\item[(c)]
\end{enumerate}

\exercisehead{2}: Pull-back and push-forward

\questionhead{}Let's check this locally
\[
\begin{aligned}
  & \phi^*(df)(X) = (df)(\phi_*X) = (df)(X^i \frac{ \partial y^j}{\partial x^i} \frac{ \partial }{ \partial y^j}  ) = X^i \frac{ \partial y^j}{ \partial x^i} \frac{ \partial f}{ \partial y^j} \text{ where } 
  & \phi_* X = X^i \frac{ \partial y^j}{ \partial x^i} \frac{ \partial }{ \partial y^j} \\ 
  & d(\phi^*f)(X) = d(f(\phi))(X) = \frac{ \partial f}{ \partial y^j} \frac{ \partial y^j}{ \partial x^i } dx^i(X) = X^i \frac{ \partial y^j}{ \partial x^i} \frac{ \partial f}{ \partial y^j}
\end{aligned}
\]
So 
\[
\boxed{ \phi^*(df) = d(\phi^* f)  } \quad \quad \, \forall \, p \in M , \, \, \forall \, X \in \mathfrak{X}(M)
\]
The big idea is that this is a showing of the \textbf{naturality} of the pullback $\phi^*$ with $d$, i.e. that this commutes:

\begin{tikzpicture}
  \matrix (m) [matrix of math nodes, row sep=2em, column sep=3em, minimum width=1em]
  {
 \Omega^1(M)  &  \Omega^1(N)   \\
C^{\infty}(M) & C^{\infty}(N)  \\  };
%  \path[-stealth]
  \path[->]
  (m-1-2) edge node [above] {$\phi^*$} (m-1-1)
%  edge node [left] {$\text{ev}_0$} (m-2-2)
%  (m-1-1) edge node [left] {$\alpha$} (m-2-1)
  (m-2-2) edge node [auto] {$d$} (m-1-2)
%  edge node [below] {$\pi_M$} (m-2-2);
  edge node [auto] {$\phi^*$} (m-2-1)
  (m-2-1) edge node [left] {$d$} (m-1-1);
\end{tikzpicture}

\questionhead{}

\[
(\phi_*)^a_{ \, \, b} := (dy^a)(\phi_*( \frac{ \partial }{ \partial x^b } ) )
\]
\[
\text{ Let } g \in C^{\infty}(N)
\]
\[
\begin{gathered}
  \phi_* \left( \frac{ \partial }{ \partial x^b} \right) g = \frac{ \partial x^b} g\phi(p) = \frac{ \partial }{ \partial x^b} g\phi x^{-1}x(p) = \frac{ \partial }{ \partial x^b}(gyy^{-1}\phi x^{-1})(x) = \\
  = \frac{ \partial }{ \partial x^b}(gy^{-1}(y\phi x^{-1}(x(p))) ) = \left. \frac{ \partial g}{ \partial y}^b \right|_y \left. \frac{ \partial y^a}{ \partial x^b} \right|_x = \frac{ \partial y^a}{ \partial x^b} \frac{ \partial g}{ \partial y^a}
\end{gathered}
\]
Then 
\[
\phi_*\left( \frac{ \partial }{ \partial x^b} \right) = \frac{ \partial y^a}{ \partial x^b} \frac{ \partial }{ \partial y^a}
\]
and so 
\[
(\phi_*)^a_{ \, \, b} = \frac{ \partial y^a}{ \partial x^b}
\]

\questionhead{}

\exercisehead{3}\textbf{:Lie derivative-the pedestrian way}

\questionhead{} While it is true that $\forall \, p \in S^2$, for $x(p) = (\theta, \varphi)$, and $(yix^{-1})(\theta,\varphi) = (y^1,y^2,y^3) \in \mathbb{R}^3$ and that, at this point $p$, $(y^1)^2/a^2 + (y^2)^2/b^2 +(y^3)^2/c^3 = 1$, this doesn't imply (EY: 20150321 I think) that, globally, it's an ellipsoid (yet).  In the familiar charts given, \\
spherical chart $(U,x) \in \mathcal{A}$ and \\
$(\mathbb{R}^3, y=\text{id}_{\mathbb{R}^3}) \in \mathcal{B}$ \\
it looks like an ellipsoid, but change to another choice of charts, and it could look something very different.  

\questionhead{}

Equip $(\mathbb{R}^3, \mathcal{O}_{\text{st}}, \mathcal{B})$ with the Euclidean metric $g$, and pullback $g$.  

Note that the pullback of the inclusion from $\mathbb{R}^3$ onto $S^2$ for the Euclidean metric is the following:
\[
i^* g\left( \frac{ \partial }{ \partial \theta^i }, \frac{ \partial }{ \partial \theta^j} \right) = g\left( i_*\frac{ \partial }{ \partial \theta^i }, i_*\frac{ \partial }{ \partial \theta^j} \right) = g\left( \frac{ \partial x^a}{ \partial \theta^i} \frac{ \partial }{ \partial x^a} , \frac{ \partial x^b}{ \partial \theta^j} \frac{ \partial }{ \partial x^b } \right) = g_{ab} \frac{ \partial x^a}{ \partial \theta^i} \frac{ \partial x^b}{ \partial \theta^j} 
\]
With $g_{ab}=\delta_{ab}$, the usual Euclidean metric, this becomes the following:
\[
g^{\text{ellipsoid}}_{ij} = \frac{ \partial x^a}{ \partial \theta^i} \frac{ \partial x^a}{ \partial \theta^j} 
\]

At this point, one should get smart (we are in the 21st century) and use some sort of CAS (Computer Algebra System). I like Sage Math (version 6.4 as of 20150322).  I also like the Sage Manifolds package for Sage Math.  

I like Sage Math for the following reasons:
\begin{itemize}
\item Open source, so it’s open and freely available to anyone, which fits into my principle of making online education open and freely available to anyone, anytime
\item Sage Math structures everything in terms of Category Theory and Categories and Morphisms naturally correspond to Classes and Class methods or functions in Object-Oriented Programming in Python and they’ve written it that way
\end{itemize}
and I like Sage Manifolds for roughly the same reasons, as manifolds are fit into a category theory framework that’s written into the Python code.  e.g.

{\small \begin{verbatim}
sage: S2 = Manifold(2, 'S^2', r'\mathbb{S}^2', start_index=1) ; print S2
sage: print S2
2-dimensional manifold 'S^2'
sage: type(S2)
<class 'sage.geometry.manifolds.manifold.Manifold_with_category'>
\end{verbatim}}

With code (I’ve provided for convenience; you can make your own as I wrote it based upon to example of $S^2$ on the sagemanifolds documentation website page), load it and do the following:

cf. \url{https://github.com/ernestyalumni/diffgeo-by-sagemnfd/blob/master/S2.sage} \\
\url{http://sagemanifolds.obspm.fr/examples.html}

{\scriptsize \begin{verbatim}
sage: load("S2.sage")
sage: U_ep = S2.open_subset('U_{ep}')
sage: eps.<the,phi> = U_ep.chart()
sage: a = var(“a”)
sage: b = var(“b”)
sage: c = var("c")
sage: inclus = S2.diff_mapping(R3, {(eps, cart): [ a*cos(phi)*sin(the), b*sin(phi)*sin(the),c*cos(the) ]} , name="inc",latex_name=r'\mathcal{i}')
sage: inclus.pullback(h).display()
inc_*(h) = (c^2*sin(the)^2 + (a^2*cos(phi)^2 + b^2*sin(phi)^2)*cos(the)^2) dthe*dthe - (a^2 - b^2)*cos(phi)*cos(the)*sin(phi)*sin(the) dthe*dphi 
- (a^2 - b^2)*cos(phi)*cos(the)*sin(phi)*sin(the) dphi*dthe + (b^2*cos(phi)^2 + a^2*sin(phi)^2)*sin(the)^2 dphi*dphi
sage: inclus.pullback(h)[2,2].expr()
(b^2*cos(phi)^2 + a^2*sin(phi)^2)*sin(the)^2
\end{verbatim}
}
A new open subset $U_{\text{ep}}$ was declared in $S^2$, a new chart $(U_{\text{ep}}, (\theta,\phi))$ was declared, the constants, $a,b,c$, were declared, and the inclusion map given in the problem
\[
y\circ \mathfrak{i} \circ x^{-1} : (\theta, \phi) \mapsto ( a\cos{\phi} \sin{\theta}, b \sin{\phi} \sin{\theta}, c\cos{\theta})
\]
Then the pullback of the inclusion map $\mathcal{i}$ was done on the Euclidean metric $h$, defined earlier in the file \begin{verbatim}S2.sage\end{verbatim}.  Then one can access the components of this metric and do, for example, \begin{verbatim}simplify_full(),full_simplify(), reduce_trig()\end{verbatim} on the expression.  

In Python, I could easily do this, and give an answer quick in LaTeX:

%{\scriptsize 
\begin{verbatim}
sage: for i in range(1,3): 
....:     for j in range(1,3):
....:         print inclus.pullback(h)[i,j].expr()
....:         latex(inclus.pullback(h)[i,j].expr() )
....:         
c^2*sin(the)^2 + (a^2*cos(phi)^2 + b^2*sin(phi)^2)*cos(the)^2
\end{verbatim}
(EY: I'll suppress the LaTeX output but this sage math function gives you LaTeX code)
%c^{2} \sin\left(\mathit{the}\right)^{2} + {\left(a^{2} \cos\left(\phi\right)^{2} + 
%b^{2} \sin\left(\phi\right)^{2}\right)} \cos\left(\mathit{the}\right)^{2}
%-(a^2 - b^2)*cos(phi)*cos(the)*sin(phi)*sin(the)
%-{\left(a^{2} - b^{2}\right)} \cos\left(\phi\right) \cos\left(\mathit{the}\right) \sin\left(\phi\right) \sin\left(\mathit{the}\right)
%-(a^2 - b^2)*cos(phi)*cos(the)*sin(phi)*sin(the)
%-{\left(a^{2} - b^{2}\right)} \cos\left(\phi\right) \cos\left(\mathit{the}\right) \sin\left(\phi\right) \sin\left(\mathit{the}\right)
%(b^2*cos(phi)^2 + a^2*sin(phi)^2)*sin(the)^2
%{\left(b^{2} \cos\left(\phi\right)^{2} + a^{2} \sin\left(\phi\right)^{2}\right)} \sin\left(\mathit{the}\right)^{2}
%
%

and so

\[
\boxed{ \begin{gathered}
 i^* g = c^{2} \sin\left(\mathit{the}\right)^{2} + {\left(a^{2} \cos\left(\phi\right)^{2} + b^{2} \sin\left(\phi\right)^{2}\right)} \cos\left(\mathit{the}\right)^{2} d\theta \otimes d\theta + \\
-2 {\left(a^{2} - b^{2}\right)} \cos\left(\phi\right) \cos\left(\mathit{the}\right) \sin\left(\phi\right) \sin\left(\mathit{the}\right) d\theta \otimes d\phi +  \\
 + {\left(b^{2} \cos\left(\phi\right)^{2} + a^{2} \sin\left(\phi\right)^{2}\right)} \sin\left(\mathit{the}\right)^{2} d\phi \otimes d\phi 
\end{gathered} }
\]

\questionhead{}

{\small
\begin{verbatim}
sage: polar_vees = eps.frame()
sage: X_1 = - sin(phi) * polar_vees[1] - cot( the ) * cos(phi) * polar_vees[2]
sage: X_2 = cos( phi ) * polar_vees[1] - cot( the ) * sin( phi) * polar_vees[2]
sage: X_3 = polar_vees[2]
sage: X_2.lie_der(X_1).display()
(cos(the)^2 - 1)/sin(the)^2 d/dphi
sage: X_3.lie_der(X_1).display()
cos(phi) d/dthe - cos(the)*sin(phi)/sin(the) d/dphi
sage: X_3.lie_der(X_2).display()
sin(phi) d/dthe + cos(phi)*cos(the)/sin(the) d/dphi
\end{verbatim}
}

Indeed, one can check on a scalar field $f_{\text{eps}} \in C^{\infty}(S^2)$:
{\small
\begin{verbatim}
sage: f_eps = S2.scalar_field({eps: function('f', the, phi ) }, name='f' )
sage: (X_1( X_2(f_eps)) - X_2(X_1(f_eps) ) ).display()
U_{ep} --> R
(the, phi) |--> -D[1](f)(the, phi)
sage: X_2.lie_der(X_1) == -X_3
True
sage: X_3.lie_der(X_1) == X_2
True
sage: X_3.lie_der(X_2) == -X_1
True
\end{verbatim}
}

\[
\Longrightarrow \boxed{ [X_i, X_j] = -\epsilon_{ijk}X_k }
\]
So $\text{span}_{\mathbb{R}} \lbrace X_1,X_2,X_3 \rbrace$ equipped with $[ \, , \, ]$ constitute a Lie subalgebra on $S^2$ (It's closed under $[ \, , \, ]$

