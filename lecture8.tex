\section{Parallel Transport \& Curvature}

\subsection{Parallelity of vector fields}
\begin{definition}
  Let $(M, \mathcal{O}, \mathcal{A}, \nabla)$ be a smooth manifold with connection $\nabla$.
  \begin{enumerate}
    \item[(1)] A vector field $X$ on $M$ is said to be \textbf{parallely transported} along a smooth curve $\gamma: \mathbb{R} \to M$ if 
      \begin{equation}
        \boxed{\nabla_{v_{\gamma}} X = 0}
      \end{equation}
    \item[(2)] A slightly weaker condition is ``\textbf{parallel}'' if, for $\mu : \mathbb{R} \to \mathbb{R}$,
      \begin{equation}
        \left(\nabla_{v_{\gamma, \gamma(\lambda)} } X\right)_{\gamma(\lambda)} = \mu(\lambda) X_{\gamma(\lambda)}
      \end{equation}
  \end{enumerate}
\end{definition}

\textbf{Note:} Even though \textbf{parallely transported} sounds like an action, it is a property.

\subsection{Autoparallely transported curves}
\begin{definition}
  A curve $\gamma: \mathbb{R} \to M$ is called \textbf{autoparallely transported} if 
  \begin{equation}
    \nabla_{v_{\gamma}}v_{\gamma} = 0
  \end{equation}
\end{definition}

\textbf{Note:} Sometimes, this curve is called an autoparallel curve. But we wish to call a curve autoparallel if $\nabla_{v_{\gamma}}v_{\gamma} = \mu v_{\gamma}$.

\subsection{Autoparallel equation}
Express $\nabla_{v_{\gamma}} v_{\gamma} = 0$ in terms of chart representation.
\begin{align*}
0 & = \left(\nabla_{v_{\gamma}} v_{\gamma}\right) \\
& = \left(\nabla_{\left(\dot{\gamma}^m_{(x)} \cibasis{x^m}\right)} \dot{\gamma}^n_{(x)} \cibasis{x^n}\right) && \text{ remember that } \gamma^m_{(x)} := x^m \after \gamma \\
& = \dot{\gamma}^m \left(\nabla_{\left(\cibasis{x^m}\right)} \dot{\gamma}^n\right) \cibasis{x^n} + \dot{\gamma}^m \dot{\gamma}^n \left(\nabla_{\left(\cibasis{x^m}\right)} \cibasis{x^n}\right) && \text{x index is understood, hence suppressed} \\
& = \dot{\gamma}^m \left(\cibasis{x^m} \dot{\gamma}^n\right) \cibasis{x^n} + \dot{\gamma}^m \dot{\gamma}^n \left(\nabla_{\left(\cibasis{x^m}\right)} \cibasis{x^n}\right) && \text{} \\
& = \dot{\gamma}^m \left(\cibasis{x^m} \dot{\gamma}^q\right) \cibasis{x^q} + \dot{\gamma}^m \dot{\gamma}^n \left(\Gamma\indices{^{q}_{nm}} \cibasis{x^q}\right) && \text{change of index in 1st term} \\
& = \left(\dot{\gamma}^m \cibasis{x^m} \dot{\gamma}^q + \dot{\gamma}^m \dot{\gamma}^n \Gamma\indices{^{q}_{nm}}\right) \cibasis{x^q} && \text{} \\
& = \left(\ddot{\gamma}^q + \dot{\gamma}^m \dot{\gamma}^n \Gamma\indices{^{q}_{nm}}\right) \cibasis{x^q} && \text{1st term is 2nd derivative by Fig. \ref{fig:L8_2ndDerivativeDerivation}}
\end{align*}

%\begin{frame}
%\begin{figure}
%\label{fig:L8_2ndDerivativeDerivation}
%\centering
%\begin{align*}
%\dot{\gamma}^m \cibasis{x^m} \dot{\gamma}^q & = \left(x^m \after \gamma\right)^\prime \cdot \partial_m\left(\dot{\gamma}^q \after x^{-1}\right)
%\end{align*}
%\caption{Second derivative of a curve}
%\end{figure}
%\end{frame}

In summary:
\begin{equation}\label{Eq:L8_autoParallelTransportChartExpression}
\boxed{\ddot{\gamma}^q_{(x)}(\lambda) + (\Gamma_{(x)})\indices{^{q}_{mn}}(\gamma(\lambda)) \dot{\gamma}^m_{(x)}(\lambda) \dot{\gamma}^n_{(x)}(\lambda) = 0}
\end{equation}
Eq. (\ref{Eq:L8_autoParallelTransportChartExpression}) is the chart expression of the condition that $\gamma$ be autoparallely transported.

%TODO after 37:07

\subsection{Torsion}
\begin{definition}
  \textbf{torsion} of a connection $\nabla$ is the $(1,2)$-\textbf{tensor field}
\begin{equation}
  T(\omega,X,Y) := \omega( \nabla_X Y - \nabla_Y X - [X,Y])
\end{equation}
\end{definition}

(Inside a cloud) 

$[X,Y]$ vector field defined by 
\[
[X,Y]f:= X(Yf) - Y(Xf)
\]

\begin{proof}
  check $T$ is $C^{\infty}$-linear in each entry

\[
\begin{gathered}
  T(\omega, fX,Y) = \omega ( \nabla_{fX} Y - \nabla_Y (fX) - [fX,Y] )
\end{gathered}
\]
\end{proof}

\begin{definition}
  A $(M, \mathcal{O}, \mathcal{A}, \nabla)$ is called torsion-free if $T=0$
\end{definition}

In a chart 
\[
\begin{aligned}
  T^i_{ \, \, ab } := T\left(dx^i , \frac{ \partial }{ \partial x^a} , \frac{ \partial }{ \partial x^b}  \right) & = dx^i ( \dots ) \\ 
  & = \Gamma^i_{ \, \, ab} - \Gamma^i_{ \, \, ba} = 2 \Gamma^i_{ \, \, [ab] }
\end{aligned}
\]

From now on, in these lectures, we only use torsion-free connections. 

\subsection{Curvature}

\begin{definition}
  \textbf{Riemann curvature} of a connection $\nabla$ is the $(1,3)$-tensor field
\begin{equation}
  \text{Riem}(\omega,Z,X,Y) := \omega( \nabla_X \nabla_Y Z - \nabla_Y \nabla_X Z - \nabla_{[X,Y]} Z)
\end{equation}
\end{definition}
\begin{proof}
  do it: $C^{\infty}$-linear in each slot.  
\end{proof}

\underline{Tutorials} $\text{Riem}^i_{ \,\, jab} = \dots $
