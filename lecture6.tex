\section{Lecture 6: Fields}

So far: 

\begin{tikzpicture}[decoration=snake]
  \matrix (m) [matrix of math nodes, row sep=2em, column sep=3em, minimum width=1em]
  {
T_pM \\ 
T_p^*M \\
\vdots \\
};
  \path[->] %,decorate={zigzag,amplitude=0.7pt, segment length=1.2mm, pre=lineto,pre length=4pt}]
  (m-1-1) edge node [left] {$\vdots$} (m-2-1)
  (m-2-1) edge node [left] {$\vdots$} (m-3-1);
\end{tikzpicture},

now

in Thought Cloud: theory of bundles

\subsection{Bundles}

\begin{definition}
  A \textbf{bundle} is a triple 
\[
E \xrightarrow{ \pi } M 
\]
$E$ smooth manifold \quad \, ``\textbf{total space}'' \\
$\pi$ smooth map (surjective)  ``projection map'' \\
$M$ smooth manifold ``base space''

\end{definition}

\underline{Example} $E = $ cylinder
$M = $ circle


\begin{definition}
  define \textbf{fibre over $p$} \\
  \phantom{define} $:= \text{preim}_{\pi}(\lbrace p \rbrace)$
\end{definition}

\begin{definition}
A \textbf{section} $\sigma$ of a bundle

\begin{tikzpicture}
  \matrix (m) [matrix of math nodes, row sep=2em, column sep=3em, minimum width=1em]
  {
    E \\
    M \\ };
  \path[->]
  (m-1-1) edge node [above] {$\phi^*$} (m-2-1)
  (m-2-1) edge [bend right=30] node [left] {$d$} (m-1-1);
\end{tikzpicture}





require $\pi \circ \sigma = \text{id}_M$
\end{definition}

Schuller says: in quantum mechanics, 
\underline{Aside}: $\psi : M \to \mathbb{C}$


\subsection{Tangent bundle of smooth manifold}

$(M,\mathcal{O},\mathcal{A})$ smooth manifold

\begin{enumerate}
\item[(a)] as a \textbf{set}
  $TM : = \dot{\bigcup}_{p\in M} T_pM$
\item[(b)] $\begin{aligned} & \quad \\
  \text{surjective } & \pi: TM \to M \\
  & X\mapsto p \end{aligned}$ the \emph{unique} point $p\in M$, $X \in T_pM$  

\underline{situation}:  $\underbrace{TM}_{\text{set}} \underbrace{ \xrightarrow{ \pi } }_{\text{surjective map }} \underbrace{M}_{\text{smooth manifold}}$

\item[(c)] Construct topology on $TM$ that is the coarsest topology such that $\pi$ (just) continuous.  (``initial topology with respect to $\pi$'').

\[
\mathcal{O}_{TM} := \lbrace \text{preim}_{\pi}(U) | \mathcal{U}\in \mathcal{O} \rbrace
\]
Show: Tutorial $\mathcal{O}_{TM}$
Schuller says this is shown in the tutorial

$(TM,\mathcal{O}_{TM})$  \\

\underline{Construction of a $C^{\infty}$-atlas on $TM$ from the $C^{\infty}$-atlas $\mathcal{A}$ on $M$. }

\[
\mathcal{A}_{TM} := \lbrace (T\mathcal{U},\xi_x  ) | (U,x) \in \mathcal{A} \rbrace
\]
where
\[
\begin{aligned}
  \xi_x : & T \mathcal{U} \to \mathbb{R}^{2\cdot\text{dim}  M } \\
  & X \mapsto (\underbrace{ (x^1 \circ \pi)(X), \dots, (x^d\circ \pi)(X) }_{(U,x)-\text{ coords of $\pi(X)$ } \, (d \text{ many } ) } , (dx^1)_{\pi(X)}(X), \dots , (dx^d)_{\pi(X)}(X)  )
\end{aligned}
\]
where $X\in T_{\pi(X)}M$ \\
\phantom{where } $X = X_{(x)}^i \left( \frac{ \partial }{ \partial x^i} \right)_{\pi(X)}$  \\
\phantom{where } $\begin{aligned} \quad  & \\
  (dx^j)_{\pi(X)}(X) &= (dx^j)_{\pi(X)} \left( X^i_{(x)}\left( \frac{ \partial }{ \partial x^i} \right)_{\pi(X)} \right) = \\
  & = X^i_{(x)}\delta_i^j = X^j_{(x)}\end{aligned}$

\underline{Write} $\xi_x^{-1} : \xi_x(TU) \subseteq \mathbb{R}^{2\text{dim}M} \to TU$
\[
(\alpha^1 , \dots , \alpha^d, \beta^1, \dots , \beta^d) := \beta^i \left( \frac{ \partial }{ \partial x^i} \right)_{ \underbrace{ x^{-1}(\alpha^1 , \dots , \alpha^d) }_{\pi(X)} }
\]

\underline{Check}: \[
\begin{gathered}
  (\xi_y \circ \xi_x^{-1})(\alpha^1 , \dots , \alpha^d, \beta^1, \dots , \beta^d) = \\
  = \xi_y \left( \beta^i \left( \frac{ \partial }{ \partial x^i} \right)_{x^{-1}(\alpha^1 , \dots , \alpha^d) } \right) \\
  = \left( \dots, (y^i \circ \pi)( \beta^m \cdot \left( \frac{ \partial }{ \partial x^m} \right)_{x^{-1}(\alpha^1 \dots \alpha^d) } )  , \dots , \dots (dy^i)_{x^{-1}(\alpha^1, \dots \alpha^d) } \left( \beta^m \left( \frac{ \partial }{ \partial x^m} \right)_{x^{-1}(\alpha^1 \dots \alpha^d) } \right) , \dots   \right) = \\
  = ( \dots , (y^i \circ x^{-1})(\alpha^1 , \dots , \alpha^d), \dots , \dots , \underbrace{ \beta^m(dy^i)_{x^{-1} (\alpha^1, \dots , \alpha^d) } \left( \left( \frac{ \partial }{ \partial x^m} \right)_{x^{-1}(\alpha^1 \dots \alpha^d) } \right)}_{ \beta^m \left( \frac{ \partial y }{ \partial x^m } \right)_{x^{-1}(\alpha^1 \dots \alpha^d)} }  ) 
\end{gathered}
\]

%
$\left( \frac{ \partial y }{ \partial x^m } \right)_{x^{-1}(\alpha^1 \dots \alpha^d)} = \partial_m (y^i \circ x^{-1} )( x\circ (x^{-1}(\alpha^1 \dots \alpha^d) ) ) = \partial_m (y^i \circ x^{-1} )( \alpha^1 \dots \alpha^d)$ smooth.  

\underline{upshot}

\[
\underbrace{TM}_{\text{smooth manifold}} \underbrace{ \xrightarrow{\pi} }_{\text{smooth map} } \underbrace{M}_{\text{ smooth manifold} }
\]
bundle, called the tangent bundle.

\end{enumerate}

\subsection{Vector fields}

\begin{definition}
  A \textbf{smooth vector field} $\chi$ is a \emph{smooth} map
\end{definition}

\subsection{The $C^{\infty}(M)$-module $\Gamma(TM)$}

\textbf{set} $\Gamma(TM) = \lbrace \chi \quad \, M \to TM | \text{ smooth section } \rbrace$
 
$(\chi \oplus \widetilde{\chi})(f) := (\chi f) + \underbrace{\widetilde{\chi}}_{C^{\infty}(M)}(f)$ \\

$(g\odot \xi)(f) := g \cdot \underbrace{ \chi }_{C^{\infty}(M)}(f)$


\underline{upshot}: set of all smooth vector fields can be made into a $C^{\infty}(M)$-module.  

\underline{Fact}: \begin{enumerate}
\item[(1)] ZF\underline{C} $\Longrightarrow $ every vector space has a basis. 
\item[(2)] no such result exists for modules.  
\end{enumerate}

This is a shame, because otherwise, we could have chosen (for any manifolds) vector fields, 
\[
\Xi_{(1)}, \dots , \Xi_{(d)} \in \Gamma(TM)
\]
and would be able to write every vector field $\Xi$
\[
\Xi = \underbrace{ f^i }_{\text{ component functions } } \cdot \Xi_{(i)}
\]

\underline{Simple counterexample}

Schuller says: Take a sphere, Morse Theorem, every smooth vector field must vanish at 2 pts. ``mustn't choose a global basis''


\underline{However}: $\begin{aligned} & \quad \\
  & \frac{ \partial }{ \partial x^i} : U \xrightarrow{ \text{ smooth }} TU \\
  & p \mapsto \left( \frac{ \partial }{ \partial x^i } \right)_p
\end{aligned}$

\subsection{Tensor fields}

so far

$\Gamma(M) = $''set of vector fields''  $C^{\infty}(M)$-module

 $\Gamma(T^*M) = $ ``covector fields'' $C^{\infty}(M)$-module

\begin{definition}
An $(r,s)$-tensor field $T$ is a multi-linear map
\[
T:\underbrace{ \Gamma(T^*M) \times \dots \times \Gamma(T^*M) }_{r} \times \Gamma(TM) \times \dots \times \Gamma(TM) \xrightarrow{ \sim } C^{\infty}(M)
\]
\end{definition}

\underline{Example}: $f\in C^{\infty}(M)$ 
\[
\begin{gathered}
  \begin{aligned} 
    df : & \Gamma(TM) \xrightarrow{ \sim } C^{\infty}(M) \\ 
    & \Xi \mapsto df(\Xi) := \Xi [f]
\end{aligned}
\end{gathered}
\]
$df$ ($0,1$)-T.F. (tensor field)

where $(\Xi f)(\underbrace{p}_{ \in M}) := \underbrace{ \Xi(p) }_{ \in T_pM}f$

can check: $df$ is $C^{\infty}-$linear
