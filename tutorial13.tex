\section*{Tutorial 13 Schwarzschild Spacetime}

EY : 20150408 I'm not sure which tutorial follows which lecture at this point.

The tutorial video is excellent itself.  Here, I want to encourage the use of CAS to do calculations.  There are many out there.  Again, I'm partial to the Sage Manifolds package for Sage Math which are both open-source and based on Python. I'll use that here.  

\exercisehead{1} \textbf{Geodesics in a Schwarzschild spacetime}

\questionhead{Write down the Lagrangian}

Load ``Schwarzschild.sage'' in Sage Math, which will always be available freely here \url{https://github.com/ernestyalumni/diffgeo-by-sagemnfd/blob/master/Schwarzschild.sage}:

{\scriptsize
\begin{verbatim}
sage: load("Schwarzschild.sage")
4-dimensional manifold 'M'
open subset 'U_sph' of the 4-dimensional manifold 'M'
Levi-Civita connection 'nabla_g' associated with the Lorentzian metric 'g' on the 4-dimensional manifold 'M'
\end{verbatim}}
and so on.

Look at the code and I had defined the Lagrangian to be \begin{verbatim}L\end{verbatim}.  To get out the coefficients of $L$ of the components of the tangent vectors to the curve, i.e. $t', r',\theta',\phi'$, denoted \begin{verbatim}tp,rp,thp,php\end{verbatim} in my .sage file, do the following:

\begin{verbatim}
sage: L.expr().coefficients(tp)[1][0].factor().full_simplify()
(2*G_N*M_0 - r)/r
sage: L.expr().coefficients(rp)[1][0].factor().full_simplify()
-r/(2*G_N*M_0 - r)
sage: L.expr().coefficients(php)[1][0].factor().full_simplify()
r^2
sage: L.expr().coefficients(thp)[1][0].factor().full_simplify()
r^2*sin(th)^2
\end{verbatim}

\questionhead{There are 4 Euler-Lagrange equations for this Lagrangian. Derive the one with respect to the function $t(\lambda)$!}

\begin{verbatim}
sage: L.expr().diff(t)
0
\end{verbatim}
This confirms that $\frac{ \partial L}{ \partial t} =0$

For $\frac{d}{d\lambda} \frac{ \partial L}{ \partial t'}$, then one needs to consider this particular workaround for Sage Math (computer technicality).  One takes derivatives with respect to declared variables (declared with var) and then substitute in functions that are dependent upon $\lambda$, and then take the derivative with respect to the parameter $\lambda$.  This does that:

{\scriptsize
\begin{verbatim}
sage: L.expr().diff( thp ).factor().subs( r == gamma1 ).subs( thp == gamma3.diff( tau ) ).subs( th == gamma3 ).diff(tau)\
....: .factor()
2*(2*cos(gamma3(tau))*gamma1(tau)*D[0](gamma3)(tau)^2 + 2*sin(gamma3(tau))*D[0](gamma1)(tau)*D[0](gamma3)(tau) 
+ gamma1(tau)*sin(gamma3(tau))*D[0, 0](gamma3)(tau))*gamma1(tau)*sin(gamma3(tau))
\end{verbatim} }

\questionhead{Show that the Lie derivative of $g$ with respect to the vector fields $K_t :=\frac{\partial}{\partial t}$}

The first line defines the vector field by accessing the frame defined on a chart with spherical coordinates and getting the time vector.  The second line is the Lie derivative of $g$ with respect to this vector field.
\begin{verbatim}
sage: K_t = espher[0]
sage: g.lie_der(K_t).display() # 0, as desired
0
\end{verbatim}

EY : 20150410 My question is this: $\forall \, X \in \Gamma(TM)$ i.e. $X$ is a vector field on $M$, or, specifically, a section of the tangent bundle, then does
\[
\mathcal{L}_Xg = 0 
\]
instantly mean that $X$ is a symmetry for $(M,g)$?  $\mathcal{L}_Xg$ is interpreted geometrically as how $g$ changes along the flow generated by $X$, and if it equals $0$, then $g$ doesn't change.  

