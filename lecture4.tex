\section{Differential Manifolds}

\begin{tikzpicture}[decoration=snake]
  \matrix (m) [matrix of math nodes, row sep=2em, column sep=3em, minimum width=1em]
  {
\gamma : \mathbb{R} & U \\
& x(U) \subseteq \mathbb{R}^d \\ 
};
  \path[->]
  (m-1-1) edge node [above] {$$} (m-1-2)
          edge node [left] {$x\circ \gamma$} (m-2-2)
  (m-1-2) edge node [right] {$x$} (m-2-2);
\end{tikzpicture}
\underline{idea}. try to ``lift'' the undergraduate notion of differentiability of a curve on $\mathbb{R}^d$ to a notion of differentiability of a curve on $M$

\underline{Problem} Can this be well-defined under change of chart?

\begin{tikzpicture}[decoration=snake]
  \matrix (m) [matrix of math nodes, row sep=4em, column sep=6em, minimum width=2em]
  {
    & y(U\cap V) \subseteq \mathbb{R}^d \\ 
\gamma : \mathbb{R} & U \cap V \neq \emptyset \\
& x(U\cap V) \subseteq \mathbb{R}^d \\ 
};
  \path[->]
  
  (m-2-1) edge node [auto] {$$} (m-2-2)
          edge node [auto] {$x\circ \gamma$} (m-3-2)
          edge node [auto] {$y\circ \gamma$} (m-1-2)
  (m-2-2) edge node [auto] {$x$} (m-3-2)
          edge node [auto] {$y$} (m-1-2)
  (m-3-2) edge [bend right=40] node [right] {$y\circ x^{-1}$} (m-1-2);
\end{tikzpicture}

$x\circ \gamma$ undergraduate differentiable (``as a map $\mathbb{R} \to \mathbb{R}^d$'')

\[
\begin{gathered}
  \underbrace{y\circ \gamma}_{\text{maybe only continuous, but not undergraduate differentiable} } =  \underbrace{ ( \overbrace{ y\circ x^{-1}}^{\mathbb{R}^d \to \mathbb{R}^d }   )}_{\text{continuous}}  \circ \underbrace{ \overbrace{ (x\circ \gamma) }^{\mathbb{R}\to \mathbb{R}^d} }_{ \text{ undergrad differentiable } }  = y \circ (x^{-1} \circ x) \circ \gamma
\end{gathered}
\]

At first sight, strategy does not work out.  

\subsection{Compatible charts}

In section 1, we used any imaginable charts on the top. mfd. $(M,\mathcal{O})$.  

To emphasize this, we may say that we took $U$ and $V$ from the \emph{maximal atlas} $\mathcal{A}$ of $(M,\mathcal{O})$.  


\begin{definition}
Two charts $(U,x)$ and $(V,y)$ of a top. mfd. are called \ding{96}-compatible if 
either
\begin{enumerate}
  \item[(a)] $U \cap V = \emptyset$
or  \item[(b)] $U\cap V \neq \emptyset$
\end{enumerate}
chart transition maps have undergraduate \ding{96} property.

EY : 20151109 e.g. since $\mathbb{R}^d \to \mathbb{R}^d$, can use undergradate \ding{96} property such as continuity or differentiability.

\[
\begin{aligned}
  & y \circ x^{-1} : x(U \cap V) \subseteq \mathbb{R}^d  \to y(U\cap V) \subseteq \mathbb{R}^d  \\
  & x\circ y^{-1} : y(U\cap V) \subseteq \mathbb{R}^d   \to x(U\cap V) \subseteq \mathbb{R}^d
\end{aligned}
\]
\end{definition}

\underline{Philosophy}: 

\begin{definition}
  An atlas $\mathcal{A}_{\text{\ding{96}}}$ is a \ding{96}-compatible atlas if any two charts in $\mathcal{A}_{\text{\ding{96}}}$ are \ding{96}-compatible.

\end{definition}

\begin{definition}
  A \textbf{\ding{96}-manifold} is a triple $(\underbrace{ M,\mathcal{O} }_{\text{top. mfd.} }, \mathcal{A}_{\text{\ding{96}}})$ \quad \, $\mathcal{A}_{\text{\ding{96}}} \subseteq \mathcal{A}_{\text{maximal}} $
\end{definition}


\begin{tabular}{ l | c  l}
\ding{96} &  undergraduate  \ding{96} &  \\
\hline
$C^0$ & $C^0(\mathbb{R}^d \to \mathbb{R}^d) =$  &  continuous maps w.r.t. $\mathcal{O}$  \\
$C^1$ & $C^1(\mathbb{R}^d \to \mathbb{R}^d) = $  &  differentiable (once) and is continuous  \\
$C^k$ & & $k$-times continuously differentiable \\
$D^k$ & & $k$-times differentiable \\
$\vdots$ & & \\
$C^{\infty}$ & $C^{\infty}(\mathbb{R}^d \to \mathbb{R}^d)$ & \\
$\mathbin{\rotatebox[origin=c]{-90}{$\supseteq$}}$ & &  \\
$C^{\omega}$ & $\exists $  multi-dim. Taylor exp.  &  \\
$\mathbb{C}^{\infty}$ & satisfy Cauchy-Riemann equations, pair-wise & 
\end{tabular}


EY : 20151109 Schuller says: $C^k$ is easy to work with because you can judge $k$-times cont. differentiability from existence of all partial derivatives \textbf{and} their continuity.  There are examples of maps that partial derivatives exist but are not $D^k$, $k$-times differentiable.  

\begin{theorem}[Whitney]
%  Any $C^{k\geq 1}$-manifold $(M,\mathcal{O}, \mathcal{A}_{C^{k\geq 1}})$  
  Any $C^{k\geq 1}$-atlas, $\mathcal{A}_{C^{k\geq 1}}$ of a topological manifold \emph{contains} a $C^{\infty}$-atlas.  

Thus we may w.l.o.g. always consider $C^{\infty}$-manifolds, ``smooth manifolds'', unless we wish to define Taylor expandibility/complex differentiability \dots
\end{theorem}

EY : 20151109 Hassler Whitney \footnote{\url{http://mathoverflow.net/questions/8789/can-every-manifold-be-given-an-analytic-structure}}

\begin{definition}
  A smooth manifold $(\underbrace{ M,\mathcal{O} }_{\text{top. mfd. } }, \underbrace{ \mathcal{A}}_{C^{\infty}-\text{atlas}} )$ 
\end{definition}

\begin{tikzpicture}
  \matrix (m) [matrix of math nodes, row sep=4em, column sep=6em, minimum width=2em]
  {
 \mathbb{R} & M   \\
&  \mathbb{R}^d \\ 
};
  \path[->]  
  (m-1-1) edge node [auto] {$\gamma$} (m-1-2)
          edge node [auto] {$x\circ \gamma$} (m-2-2)
  (m-1-2) edge node [auto] {$x$} (m-2-2);
\end{tikzpicture}
EY: 20151109 Schuller was explaining that the trajectory is real in $M$; the coordinate maps to obtain coordinates is $x\circ \gamma$

\subsection{Diffeomorphisms}

$M \xrightarrow{ \phi } N$

If $M,N$ are naked sets, the structure preserving maps are the bijections (invertible maps).  

e.g. $\lbrace 1,2,3 \rbrace \to \lbrace a,b \rbrace$

\begin{definition}
  $M \cong_{\text{set}} N$ (set-theoretically) isomorphic if $\exists \, $ bijection $\phi : M \to N$
\end{definition}

\underline{Examples}.  $\mathbb{N} \cong_{\text{set}} \mathbb{Z}$ \\
$\mathbb{N} \cong_{\text{set}} \mathbb{Q}$  (EY: 20151109 Schuller says from diagonal counting)\\
$\mathbb{N} \cancel{\cong_{\text{set}}} \mathbb{R}$

Now $(M, \mathcal{O}_M) \cong_{\text{top}} (N,\mathcal{O}_N)$ (topl.) isomorphic $=$ ``homeomorphic'' $\exists \, $ bijection $\phi : M \to N$  \\
\phantom{ \quad \quad \, } $\phi, \phi^{-1}$ are continuous.  

$(V,+,\cdot) \cong_{\text{vec}} ( W,+_w,\cdot_w)$ (EY: 20151109 vector space isomorphism) if \\
$\exists \, \text{ bijection } \phi : V \to W$ linearly

\underline{finally}

\begin{definition}
  Two $C^{\infty}$-manifolds \\
  $(M,\mathcal{O}_M, \mathcal{A}_M)$ and $(N,\mathcal{O}_N, \mathcal{A}_N)$ are said to be \textbf{diffeomorphic} if $\exists \, $ bijection $\phi : M \to N$ s.t. 
\[
\begin{aligned} & \phi : M \to N \\
  & \phi^{-1} : N \to M \end{aligned}
    \]
are both $C^{\infty}$-maps

\begin{tikzpicture}
  \matrix (m) [matrix of math nodes, row sep=4em, column sep=6em, minimum width=2em]
  {
 \mathbb{R}^d & \mathbb{R}^e   \\
M \supseteq U &  V\subseteq N   \\ 
\mathbb{R}^d & \mathbb{R}^e \\
};
  \path[->]  
  (m-1-1) edge node [auto] {$\widetilde{y} \circ \phi \circ \widetilde{x}^{-1}$} (m-1-2)
  (m-2-1) edge node [auto] {$\widetilde{x}$} (m-1-1)
          edge node [auto] {$\phi$} (m-2-2)
          edge node [auto] {$x$} (m-3-1)
  (m-3-1) edge node [auto] {$ \substack{ y\circ \phi \circ x^{-1} \\ 
 \text{ undergraduate } C^{\infty} }$} (m-3-2)
          edge [bend left=50] node [auto] {$C^{\infty}$} (m-1-1)
  (m-2-2) edge node [auto] {$\widetilde{y}$} (m-1-2) 
          edge node [auto] {$y$} (m-3-2)
  (m-3-2) edge [bend right=50] node [auto] {$$} (m-1-2);
\end{tikzpicture}


\end{definition}

\begin{theorem}
  $\# = $ number of $C^{\infty}$-manifolds one can make out of a given $C^0$-manifolds (if any) - up to diffeomorphisms.  

\begin{tabular}{l | c r }
  $\text{dim}M$ &  $\#$ &  \\
  \hline
  1  & 1  & Morse-Radon theorems \\
 2  & 1  & Morse-Radon theorems \\
 3 & 1  & Morse-Radon theorems \\
4 & uncountably infinitely many & \\
5 &   finite  & surgery theory \\
6 &  finite & surgery theory \\
\vdots & finite & surgery theory
\end{tabular}

\end{theorem}

EY : 20151109 cf. \url{http://math.stackexchange.com/questions/833766/closed-4-manifolds-with-uncountably-many-differentiable-structures}  \\
\href{http://www.maths.ed.ac.uk/~aar/papers/scorpan.pdf}{The wild world of 4-manifolds}
