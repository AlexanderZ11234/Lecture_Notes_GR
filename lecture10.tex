\section{Metric Manifolds}
We establish a structure on a smooth manifold that allows one to assign vectors in each tangent space a length (and an angle between vectors in the same tangent space). From this structure, one can then define a notion of length of a curve. Then we can look at shortest curves (which will be called \textbf{geodesics}).

Requiring then that the shortest curves coincide with the straight curves (w.r.t. $\nabla$) will result in $\nabla$ being determined by the metric structure $g$. $\nabla$, in turn determines the curvature given by $Riem$. Thus
\[
g  \overset{\substack{\text{straight} = \text{shortest/} \\ \text{longest/ stationary curves} \\ T =0}}{\rightsquigarrow}  \nabla \rightsquigarrow Riem
\]

\subsection{Metrics}
\begin{definition}
  A metric $g$ on a smooth manifold $\mfd$ is a $(0,2)$-tensor field satisfying
\begin{enumerate}[(i)]
\item \textbf{symmetry: }$g(X,Y) = g(Y,X) \quad \forall \, X, Y \text{ vector fields}$
\item \textbf{non-degeneracy: }the \underline{musical map}
\begin{align*}
  \text{``flat''} \, \,  \flat : \Gamma(TM) & \to \Gamma(T^*M) \\ 
  X \mapsto \flat(X) \\
\text{ where } & \flat(X)(Y):= g(X,Y) \\
  & \flat(X) \in \Gamma(T^*M)
\end{align*}
In thought bubble: $\flat(X) = g(X,\cdot)$

\dots is a $C^{\infty}$-isomorphism in other words, it is invertible.
\end{enumerate}
\end{definition}

\underline{Remark}: $(\flat(X))_a$ \quad \, or \\
$X_a$ \\
$(\flat(X))_a := g_{am} X^m$

Thought bubble: $\flat^{-1} = \sharp$

$\flat^{-1}(\omega)^a := g^{am}\omega_m$ \\
$\flat^{-1}(\omega)^a := (g^{``-1''})^{am}\omega_m \implies$ not needed. (all of this is not needed)

\begin{definition}
The $(2,0)$-tensor field $g^{``-1''}$ with respect to a metric $g$ is the symmetric
\begin{align*}
  g^{``-1''} : \Gamma(T^*M) \times \Gamma(T^*M) \xrightarrow{ ~ } C^{\infty}(M) \\
  (\omega, \sigma) \mapsto \omega(\flat^{-1}(\sigma)) \quad \quad \, \flat^{-1}(\sigma) \in \Gamma(TM))
\end{align*}

\underline{chart}: $g_{ab} = g_{ba} \quad \quad (g^{-1})^{am} g_{mb} = \delta^a_b$
\end{definition}

\underline{Example}: Consider $(S^2, \mathcal{O}, \A)$ and the chart $(U,x)$
\begin{align*}
\varphi \in (0,2\pi), & \quad \quad \theta \in (0,\pi)
\end{align*}

Define the metric \\
\[
g_{ij}(x^{-1}(\theta,\varphi)) = \left[ \begin{matrix} R^2 & 0 \\
    0 & R^2\sin^2{\theta} \end{matrix} \right]_{ij}
\]
$R \in \R^+$

``the metric of the \underline{round sphere of radius $R$}''

\subsection{Signature}
Linear algebra: \quad \quad \, $\begin{aligned} & A\indices{^a_{m}}v^m = \lambda v^a & \quad \quad \quad \, \left(\begin{matrix} \lambda_1 & & 0 \\
    & \ddots & \\ 
    0 & & \lambda_n \end{matrix} \right) \\
  & g_{am} v^m = \lambda \cdot v^a ? \rightsquigarrow  & \quad \quad \quad \, \left( \begin{matrix} 
    1        &   &    &        &    &   &        & \\
    & \ddots &   &    &        &    &   &        & \\
    &        & 1 &    &        &    &   &        & \\
    &        &   & -1 &        &    &   &        & \\
    &        &   &    & \ddots &    &   &        & \\
    &        &   &    &        & -1 &   &        & \\
    &        &   &    &        &    & 0 &        & \\
    &        &   &    &        &    &   & \ddots & \\
    &        &   &    &        &    &   &        & 0 \end{matrix} \right)
\end{aligned}$

$(1,1)$ tensor has eigenvalues \\
$(0,2)$ has \underline{signature} $(p,q)$ (well-defined)

$\left. \begin{aligned}
  (+++) \\
  (++-) \\
  (+--) \\
  (---) \end{aligned} \right\rbrace$ $d+1$ if $p+q = \text{dim}V$

\begin{definition} A metric is called \textbf{Riemannian} if its signature is $(++ \dots +)$, and \textbf{Lorentzian} if it is $(+-\dots -)$.
\end{definition}

\subsection{Length of a curve}
Let $\gamma$ be a smooth curve. Then we know its veloctiy $v_{\gamma,\gamma(\lambda)}$ at each $\gamma(\lambda) \in M$.

\begin{definition}
On a Riemannian metric manifold $(M, \mathcal{O}, \A, g)$, the \textbf{speed} of a curve at $\gamma(\lambda)$ is the number 
\begin{equation}
\boxed{s(\lambda) = (\sqrt{g(v_{\gamma}, v_{\gamma})})_{\gamma(\lambda)}}
\end{equation}
\end{definition}

(I feel the need for speed, then I feel the need for a metric)

\underline{Aside}: $[v^a] = \frac{1}{T}$ \\
\phantom{Aside:} $[g_{ab}] = L^2 $ \\
\phantom{Aside:} $[\sqrt{g_{ab}v^av^b}] = \sqrt{ \frac{L^2}{T^2}} = \frac{L}{T}$

\begin{definition}
Let $\gamma:(0,1) \to M$ a smooth curve. Then the \textbf{length of $\gamma$}, $L[\gamma] \in \R$ is the number
\begin{equation}
\boxed{L[\gamma] := \int_0^1 d\lambda s(\lambda) = \int_0^1 d\lambda \sqrt{ (g(v_{\gamma}, v_{\gamma}))_{\gamma(\lambda)}}}
\end{equation}
\end{definition}

F. Schuller: ``velocity is more fundamental than speed, speed is more fundamental than length''

\textbf{Example:} Reconsider the round sphere of radius $R$. Consider its equator:
\begin{align*}
\theta(\lambda) := (x^1 \after \gamma)(\lambda) = \frac{\pi}{2}, & \quad \varphi(\lambda) := (x^2 \after \gamma)(\lambda) = 2\pi \lambda^3 \\
\implies \theta'(\lambda) = 0, & \quad \varphi'(\lambda) = 6\pi\lambda^2
\end{align*}

On the same chart $g_{ij} = \left[ \begin{matrix} R^2 & \\
    & R^2 \sin^2{\theta} \end{matrix} \right]$

Do everything in this chart
\begin{align*}
L[\gamma] & = \int_0^1 d\lambda \sqrt{g_{ij}(x^{-1}(\theta(\lambda), \varphi(\lambda)))(x^i \after \gamma)'(\lambda)(x^j \after \gamma)'(\lambda)} \\
& = \int_0^1 d\lambda \sqrt{R^2 \cdot 0 + R^2\sin^2{(\theta(\lambda))} 36 \pi^2 \lambda^4} \\
& = 6\pi R \int_0^1 d\lambda \lambda^2 = 6\pi R [\frac{1}{3} \lambda^3]^1_0 = 2\pi R
\end{align*}

\begin{theorem}
$\gamma : (0,1) \to M$ and $\sigma :(0,1) \to (0,1)$ smooth bijective and \underline{increasing} ``reparametrization'' \\
$L[\gamma] = L[\gamma \after \sigma]$
\end{theorem}

\begin{proof}
  in Tutorials
\end{proof}

\subsection{Geodesics}
\begin{definition}
A curve $\gamma : (0,1) \to M$ is called a \textbf{geodesic} on a Riemannian manifold $(M, \mathcal{O}, \A, g)$ if it is a stationary curve with respect to a length functional $L$.
\end{definition}

Thought bubble: In classical mechanics, deform the curve a little, $\epsilon$ times this deformation, to first order, it agrees with $L[\gamma]$.

\begin{theorem}
$\gamma$ is geodesic iff it satisfies the Euler-Lagrange equations for the Lagrangian
\end{theorem}
\begin{align*}
\mathcal{L} : & TM \to \R \\
& X \mapsto \sqrt{g(X,X)}
\end{align*}
In a chart, the Euler Lagrange equations take the form:
\[
\left(\cibasis[\mathcal{L}]{\dot{x}^m}\right)^{\cdot} - \cibasis[\mathcal{L}]{x^m} = 0 
\]
F.Schuller: this is a chart dependent formulation

here: 
\[
\mathcal{L}(\gamma^i, \dot{\gamma}^i) = \sqrt{g_{ij}(\gamma(\lambda)) \dot{\gamma}^i(\lambda) \dot{\gamma}^j(\lambda)}
\]
Euler-Lagrange equations:
\begin{align*}
\cibasis[\mathcal{L}]{\dot{\gamma}^m} = \frac{1}{\sqrt{\dots}} g_{mj}(\gamma(\lambda)) \dot{\gamma}^j(\lambda) \\
\left(\cibasis[\mathcal{L}]{\dot{\gamma}^m}\right)^{\cdot} = \left(\frac{1}{\sqrt{\dots}} \right)^{\cdot} g_{mj}(\gamma(\lambda)) \cdot \dot{\gamma}^j(\lambda) + \frac{1}{\sqrt{\dots}} \left(g_{mj}(\gamma(\lambda)) \ddot{\gamma}^j(\lambda) + \dot{\gamma}^s(\partial_s g_{mj}) \dot{\gamma}^j(\lambda) \right)
\end{align*}
Thought bubble: reparametrize $g(\dot{\gamma}, \dot{\gamma}) = 1$ (it's a condition on my reparametrization)

By a clever choice of reparametrization $(\frac{1}{\sqrt{\dots}})^{\cdot} = 0$
\[
\cibasis[\mathcal{L}]{\gamma^m} = \frac{1}{2\sqrt{\dots}} \partial_m g_{ij}(\gamma(\lambda)) \dot{\gamma}^i(\lambda) \dot{\gamma}^j(\lambda)
\]
putting this together as Euler-Lagrange equations:
\begin{align*}
g_{mj} \ddot{\gamma}^j + \partial_s g_{mj} \dot{\gamma}^s \dot{\gamma}^j - \frac{1}{2} \partial_m g_{ij} \dot{\gamma}^i \dot{\gamma}^j = 0 \\
\ddot{\gamma^q} + (g^{-1})^{qm}(\partial_i g_{mj} - \frac{1}{2} \partial_m g_{ij}) \dot{\gamma}^i \dot{\gamma}^j = 0 && (\text{multiply on both sides }(g^{-1})^{qm}) \\
\boxed{\ddot{\gamma^q} + (g^{-1})^{qm}\frac{1}{2} (\partial_i g_{mj} + \partial_j g_{mi} - \partial_m g_{ij}) \dot{\gamma}^i \dot{\gamma}^j = 0}
\end{align*}
geodesic equation for $\gamma$ in a chart.  
\[
\boxed{(g^{-1})^{qm}\frac{1}{2} (\partial_i g_{mj} + \partial_j g_{mi} - \partial_m g_{ij} ) =: \ccf{q}{ij}(\gamma(\lambda))
}
\]
Thought bubble: $\left(\cibasis[\mathcal{L}]{\xi_x^{a+\text{dim}M}} \right)^{\cdot}_{\sigma(x)} - \left(\cibasis[\mathcal{L}]{xi^a_x} \right)_{\sigma(x)} = 0$

\begin{definition}
\textbf{Christoffel symbol} ${\,}^{\text{L.C.}}\Gamma$ are the connection coefficient functions of the so-called Levi-Civita connection ${\,}^{\text{L.C.}}\nabla$
\end{definition}
We usually make this choice of $\nabla$ if $g$ is given.  

$(M, \mathcal{O}, \A, g) \to (M, \mathcal{O}, \A, g, {\,}^{\text{L.C.}}\nabla)$

\underline{abstract way}: $\nabla g = 0$ and $T = 0$ (torsion) \\
$\Longrightarrow \nabla = {\,}^{\text{L.C.}}\nabla$

\begin{definition}
\begin{enumerate}[(a)]
\item The \textbf{Riemann-Christoffel curvature} is defined by 
\begin{equation}
\boxed{R_{abcd} := g_{am}R\indices{^m_{bcd}}}
\end{equation}

\item \textbf{Ricci}
\begin{equation}
\boxed{R_{ab} = R\indices{^m_{amb}}}
\end{equation}
Thought bubble: with a metric, ${\,}^{\text{L.C.}}\nabla$

\item (Ricci) scalar curvature:
\begin{equation}
\boxed{R = g^{ab} R_{ab}}
\end{equation}

Thought bubble: ${\,}^{\text{L.C.}}\nabla$
\end{enumerate}
\end{definition}

\begin{definition}
\textbf{Einstein curvature} of $(M, \mathcal{O}, \A, g)$ is defined as
\begin{equation}
\boxed{G_{ab} := R_{ab} - \frac{1}{2} g_{ab} R}
\end{equation}
\end{definition}

\underline{Convention}: $g^{ab} := (g^{``-1''})^{ab}$

F. Schuller: these indices are not being pulled up, because what would you pull them up with

(student) Question: Does the Einstein curvature yield new information? \\
Answer: \\
$g^{ab} G_{ab} = R_{ab} g^{ab} - \frac{1}{2} g_{ab} g^{ab} R = R - \delta^a_a R = R - \frac{1}{2} \text{dim}M \, R = (1- \frac{d}{2}) R$
